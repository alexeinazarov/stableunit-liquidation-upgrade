%---------------------------------------------------------------------------%
%  A Cascade-Resilient Liquidation Architecture for CDP-Backed Stablecoins  %
%---------------------------------------------------------------------------%
\documentclass[11pt]{article}

% --------------------  PACKAGES  --------------------
\usepackage[T1]{fontenc}
\usepackage{lmodern}

% Load microtype *before* hyperref for better results
\usepackage{microtype}
\usepackage{float}
\floatstyle{ruled}
\newfloat{listing}{htbp}{lop}
\floatname{listing}{Listing}

\usepackage{xcolor}
\usepackage{amsmath,amssymb,amsfonts,mathtools}
\usepackage{booktabs,siunitx}
\sisetup{detect-all,group-separator={,}}
\usepackage{colortbl}
\usepackage{enumitem}
\usepackage[table]{xcolor}
\usepackage{bookmark}
\usepackage{xcolor}
\definecolor{darkgreen}{HTML}{006400}

\usepackage{listings}
\lstset{
  basicstyle   =\ttfamily\small,
  frame        =single,
  columns      =flexible,
  breaklines   =true,
  rulecolor    =\color{black!50},
  tabsize      =2
}

\usepackage{graphicx}
\usepackage[margin=1in]{geometry}

\usepackage{hyperref}
\hypersetup{
  colorlinks,
  linkcolor=blue!60!black,
  citecolor=blue!60!black,
  urlcolor =blue!60!black
}
\pdfstringdefDisableCommands{\let\secref\ref}

% clickable §<number> references; doesn't work here
\newcommand{\secref}[1]{\hyperref[#1]{\S\ref*{#1}}}

\usepackage{tocloft}
\setlength{\cftbeforesecskip}{6pt}
\setlength{\cftbeforesubsecskip}{2pt}
\renewcommand{\cftsecfont}{\normalfont}
\renewcommand{\cftsubsecfont}{\normalfont\itshape}

% For theorems and proofs
\usepackage{amsthm}

% Theorem-like environments
\newtheorem{theorem}{Theorem}[section]
\newtheorem{corollary}[theorem]{Corollary}
\newtheorem{lemma}[theorem]{Lemma}

% Proof environment is auto-enabled by amsthm

% For code listings
\usepackage{minted}
\usepackage{xcolor}
\definecolor{codegray}{gray}{0.97}


% --------------------  BIBLIOGRAPHY (biblatex + biber) --------------------
\usepackage[
  backend=biber,
  style=authoryear,
  doi=true,
  url=true,
  isbn=false
]{biblatex}



\addbibresource{refs.bib}
\nocite{*}  % Include all refs even if not cited explicitly

% --------------------  TITLE  --------------------
\title{\bfseries
A Cascade-Resilient Liquidation Architecture\\[-2pt]
for CDP-Backed Stablecoins}
\author{Alexey Nazarov%
  \thanks{Professor at Sorbonne Université; Université Paris 8; Paris-Diderot; and IPSA. The author thanks Alex Lebed for insightful discussions.}
}
\date{\today}


% --------------------  DOCUMENT  --------------------
\begin{document}
\maketitle

\begin{abstract}
Liquidation engines constitute the primary line of defence for over-collateralised
stablecoins. We combine square-root market-impact theory, central-bank stress
tests, and reliability-engineering cascade models to design an \emph{institution-grade}
safety net that (i) suppresses fire-sale feedback, (ii) bounds bad debt at the
\SI{99.9}{\percent}~VaR level, and (iii) remains fully on-chain auditable.
An indicative backtest of the ETH flash crash on 11~December~2024 shows an
\textbf{84.6\%} reduction in bad debt versus a \textsc{5\%} fixed-spread
liquidator while cutting peak price impact by \textbf{10.4} percentage points.
\end{abstract}

\clearpage
\pdfbookmark{Contents}{toc}
\tableofcontents
\clearpage


%=====================================================================%  
\section{From Specification to Formal Upgrade}

The StableUnit protocol \parencite{stableunit2025spec} provides a heuristic liquidation mechanism. This paper formalizes its design and introduces risk-aware, cascade-resilient upgrades. The table below presents a direct mapping between specification phrases and improvements introduced here.
%=====================================================================%  
\rowcolors{2}{gray!10}{white}
\begin{table}[ht]
\centering
\caption{Design responses to fragile elements in the StableUnit specification.}
\label{tab:trigger-map}

\begin{tabular}{@{}p{0.5\linewidth} p{0.45\linewidth}@{}}
\toprule
\textbf{StableUnit Specification} & \textbf{Improved Design (This Work)} \\
\midrule
“Liquidations are done through market sales.” & \secref{sec:tranch} –- \secref{sec:auction}: replaced with liquidity-aware tranching and sealed-bid auctions to reduce price impact. \\
“Price has a discount that gets bigger per block.” & \secref{sec:reserve}: replaced by fixed 99.9\% VaR haircut $(1 - \delta)$ to avoid spiral discounting. \\
“Liquidator has 60 seconds to sell the collateral.” & \secref{sec:tranch}: 60-second window formalised using a rotating 21-slot queue with congestion-aware timing and slashing. \\
“Trader bot monitors isLiquidateble().” & \secref{sec:bots}: architecture explicitly splits into \textit{Trigger bot} and \textit{Fill bot}, clarifying role separation. \\
“Whitelist of stablecoins … to exchange the collateral.” & \secref{sec:reserve}: wrapped in zk-attested three-oracle median to defend against price feed manipulation. \\
“Risks … cascade liquidation … chain reaction.” & \secref{sec:stress}: introduces $\kappa\sigma$ stress detection and GARCH-based volatility triggers. \\
“Bot uses AAVE’s Flashloans.” & \secref{sec:tranch}: tranche size $Q_i$ adapts to $\text{Depth}_{1\%}$, mitigating flash-loan risk if liquidity vanishes. \\
“Caller receives 0.1\% of the collateral.” & \secref{sec:auction}-–\secref{sec:keeper}: replaced by 2nd-price auction with 1\% refundable bond and a diminishing-marginal bounty function, reducing protocol costs and limiting whale rewards. \\
“No other liquidator has the right to buy … 60 seconds.” & \secref{sec:auction} + Code listing: commit-reveal removes gas-sniping risk even after exclusivity window. \\
“Protocol is in no rush to exchange for USD Pro.” &  \secref{sec:buyback}: formalised buffer mechanism swaps at reserve price and burns USD Pro post-conversion. \\
\bottomrule
\end{tabular}
\end{table}


%=====================================================================%  
\section{System Overview}
%=====================================================================%  
The liquidation engine is upgraded from a monolithic, discount-driven process into a modular architecture that combines stress detection, depth-aware liquidation, and auction-theoretic efficiency.

At its core, the protocol must decide—within a single EVM call—whether to:
\begin{itemize}
  \item \textbf{(a)} execute a spot sale via tranche slicing with tight slippage control, or
  \item \textbf{(b)} escalate into a capital-preserving batch auction under cascade stress.
\end{itemize}

Each upgrade responds directly to heuristic or fragile elements in the original StableUnit specification (see Table~\ref{tab:trigger-map}) and is anchored in risk metrics like volatility, depth, and Value-at-Risk.

\vspace{1em}
\noindent
\textbf{Core Components:}
\begin{enumerate}[label=\textbf{\arabic*.},wide, labelindent=0pt]
  \item \textbf{Stress Regime Detector} (\secref{sec:stress}): monitors realized volatility and spot/TWAP deviation to trigger auction mode only during cascading conditions.

  \item \textbf{Liquidity-Adaptive Tranching} (\secref{sec:tranch}): converts collateral into execution-sized slices based on real-time AMM depth and volatility, reducing slippage and flash-loan exposure.

  \item \textbf{Sealed-Bid Vickrey Auction} (\secref{sec:auction}): replaces priority gas wars with sealed-bid commitment, bid-bond deterrence, and second-price clearing—eliminating MEV and griefing.

  \item \textbf{Trigger–Fill Separation \& Keeper Queue} (\secref{sec:bots}): formalizes the bot logic from the original spec as a ve-style prioritized 21-slot keeper ring with congestion-aware timing.

  \item \textbf{Dynamic Keeper Bounties} (\secref{sec:keeper}): replaces fixed 0.1\% liquidator fees with a diminishing-marginal payout curve, aligning incentives across vault sizes while capping micro-vault APRs.

  \item \textbf{VaR-Driven Reserve Price} (\secref{sec:reserve}): implements a median-based oracle mechanism with 99.9\% Value-at-Risk haircut, protecting against fire-sale spirals and oracle deviations.

  \item \textbf{Contagion Heat-Map \& Cooldowns} (\secref{sec:contagion}): broadcasts real-time liquidation pressure using $k$-shell decomposition; peers may initiate self-pauses based on outer-shell exposure.

  \item \textbf{Buyback Pipeline} (\secref{sec:buyback}): buffers filled stablecoins, converts to USD Pro only at fair market conditions, and burns USD Pro to close debt—improving capital efficiency and transparency.

  \item \textbf{Governance \& Param Control} (\secref{sec:governance}): central parameters \((\kappa, Y, \psi, \delta)\) reside in a timelocked config module; supporting notebooks are pinned to Arweave and changes are publicly emitted via \texttt{ParamChange(hash)} events.
\end{enumerate}

\noindent
Together, these components offer a cascade-resilient, auction-driven liquidation protocol that meets the demands of volatile market conditions without sacrificing decentralization or composability.


%=====================================================================%  
\section{Stress-regime detection}
\label{sec:stress}
%=====================================================================%  

\subsection{Deterministic core}

Stress is declared if
\begin{equation}\label{eq:stress-core}
\text{stress} =
\left( \left| P_{\text{oracle}} - \text{TWAP}_{24\mathrm{h}} \right| > \kappa \widehat{\sigma}_{t} \right)
\;\lor\;
\left( \widehat{\sigma}_{t} > Y \right),
\qquad \kappa = 1.65.
\end{equation}

Here \( \widehat{\sigma}_t \) denotes the empirical 24-hour realized volatility \parencite{andersen2003realized}, computed as
\begin{equation}
\widehat{\sigma}_{t} = \sqrt{\frac{1}{n} \sum_{i=1}^{n} r_{t-i}^2}, 
\qquad
r_t = \ln\left( \frac{P_t}{P_{t-1}} \right),
\end{equation}
where \( r_t \) is the log return and \( P_t \) is the mid-price at time \( t \).


\parencite{tian2025defi} show $\kappa = 1.65$ minimises false positives on
ETH/USD (Figure 6).

\subsection{Adaptive forecast (optional)}
Replace $\sigma_{\text{real}}$ by a \emph{GARCH(1,1)} forecast,
\[
\sigma_{t+1}^2 = \omega + \alpha \epsilon_t^2 + \beta \sigma_t^2,
\]
pushed via oracle every five minutes.  
Appendix C of \citeauthor{tian2025defi} reports a
\SI{30}{\percent} reduction in false negatives.

%=====================================================================%
\section{Liquidity-Adaptive Tranching}
\label{sec:tranch}
%=====================================================================%
To reduce slippage, flash-loan exposure, and MEV vulnerability, the protocol divides liquidation inventory into dynamic tranches rather than executing full spot sales in a single transaction. Each tranche size \( Q_i \) adapts to both real-time liquidity and market volatility.

The tranche formula is:
\begin{equation}\label{eq:tranche}
Q_i = \min \left[ Q_{\text{rem}},\;
                  \psi\,\text{Depth}_{1\%}\,e^{-\gamma \widehat{\sigma}_t} \right],
\qquad \psi = 0.12,\quad \gamma = 2.1.
\end{equation}

Here, \( \widehat{\sigma}_t \) denotes as in \secref{sec:stress} the empirical 24-hour realized volatility at time \( t \), and \( \text{Depth}_{1\%} \) refers to the executable on-chain liquidity within a \(\pm 1\%\) band around the mid-price \( P_t \). Formally:
\[
\text{Depth}_{1\%} := \min \left(
\int_{P_t \cdot 0.99}^{P_t \cdot 1.01} \text{SellQty}(p)\,dp,\;
\int_{P_t \cdot 0.99}^{P_t \cdot 1.01} \text{BuyQty}(p)\,dp
\right),
\]
where \( \text{SellQty}(p) \) and \( \text{BuyQty}(p) \) represent the quantity available at price level \( p \) from automated market makers (AMMs).\footnote{On Uniswap V3, this corresponds to simulating a swap via the \texttt{quoter.quoteExactInputSingle()} method across the \(\pm1\%\) price range and summing executable ticks on both sides of the pool \parencite{uniswap-v3-quoter-docs}.}

As shown in \textcite{almgren2000optimal}, execution cost scales with \( \sqrt{Q} \), making tranche-based execution significantly more efficient. Quarter-slicing reduces expected slippage by nearly 50\% in typical AMM conditions.  
Reinforcement learning (RL)-based dynamic adjustments \parencite{zhang2023adaptive} are theoretically compatible and may be proposed for deployement.  
This module also aligns with the cascade model taxonomy proposed in \textcite{zhao2025reliability}, supporting fault-tolerant execution under stress conditions.




%=====================================================================%  
\section{Auction Mechanics and Keeper Incentives}
\label{sec:auction}
%=====================================================================%  

Recent empirical work shows a clear shift in DeFi from fixed-spread mechanisms \parencite{aave-liquidation-docs, compound-liquidation-docs}—toward \textbf{auction-based liquidations}, exemplified by MakerDAO’s \textit{tend–dent} auction \parencite{makerdao-auctions-docs}. These auctions yield \textbf{approximately 80\% lower on-chain price impact} and significantly reduce contagion risk by incentivizing liquidator competition. Smart Value Recapture (SVR) \parencite{chainlink2025svr} complements this trend by integrating oracle-driven pre-auctions to internalize MEV during liquidation events.

We formalize the liquidation mechanism as a sealed-bid, second-price auction \parencite{vickrey1961counterspeculation}, in which truth-telling is a dominant strategy \parencite{myerson1994bayesian}, to minimize winner’s curse and MEV risk\textcolor{darkgreen}{\footnote{A full implementation of the Vickrey wrapper \texttt{ClipVickrey} is provided in Appendix~C.}}. Bids are submitted by keeper bots off-chain and revealed on-chain after a delay; the protocol then selects the highest eligible bid and settles at the second-highest price. A 1\% refundable bid bond deters griefing and MEV sniping \parencite{tian2025defi}.

Empirical simulations suggest this auction design reduces liquidation costs by \SI{22}{\percent} compared to fixed-spread execution. To prevent excessive rent extraction, keeper reward is capped at:
\begin{equation}
\text{fee}_{\max} = 0.5\, \delta_{\text{VaR}}\, V
\end{equation}

where \(\delta_{\text{VaR}}\) is a Value-at-Risk-derived haircut defined in \secref{sec:reserve} and \( V \) is the collateral value.

\begin{itemize}[leftmargin=*]
  \item \textbf{Format} — sealed-bid, second-price (Vickrey-$\alpha$).
  \item \textbf{Bid bond} — \SI{1}{\percent} refundable.
  \item \textbf{Reveal window} — $k{=}10$ blocks (\SI{120}{\second}).
\end{itemize}

Let $\mathcal{B} = \{b_1, \dots, b_n\}$ be the set of valid bids. The clearing price is
\[
P_{\text{clear}} = \max\left\{ b_j \in \mathcal{B} \,|\, b_j \geq P_{\text{reserve}},\; b_j \leq b_{(k)} \right\},
\]
where $b_{(k)}$ is the $k$-th highest bid. This bounds the winner's curse risk.


%=====================================================================%  
\section{Trigger–Fill Separation and Keeper Prioritisation}
\label{sec:bots}
%=====================================================================%  
Inspired by StableUnit’s dual-bot architecture \parencite{stableunit2025spec}, we formalize the protocol's separation of duties:

\begin{enumerate}[label=\textbf{\arabic*.},wide, labelindent=0pt]
  \item \textbf{Trigger bot} — monitors CDPs via \texttt{isLiquidatablePosition} and initiates liquidation.
  \item \textbf{Fill bot} — a keeper from a ranked staking queue, with a \SI{60}{\second} exclusive execution window.
\end{enumerate}

Priority is assigned using SuDAO's vote-escrow (ve-style) staking model, where keeper eligibility increases with lock duration and stake amount. 
Each CDP is hashed to one of \textcolor{darkgreen}{$S_{t}$} privileged keeper slots\footnote{\textcolor{darkgreen}{%
\(S_{t}\!\in\!\{2,4,8,16,32,64\}\) is the \emph{epoch-indexed slot cardinality} held in \texttt{RiskConfig.SLOT\_COUNT}.  
A DAO vote (or the adaptive controller) may update \(S_{t}\) at the next epoch boundary, leaving earlier liquidations unaffected. Empirical guidance: \(\,2{\text{–}}4\) slots for proof-of-concept deployments; \(8{\text{–}}16\) once TVL surpasses \(\sim\$50\) M; \(32{\text{–}}64\) in high-frequency, HFT-keeper environments.  Slot routing is the constant-time hash  
\(\mathrm{slotId}=\mathtt{keccak256(vaultId)}\bmod S_{t}\).  
Because the queue is stored as a sparse mapping \texttt{mapping(uint256~=>~Slot)}, unclaimed indices consume \emph{zero} storage gas, so enlarging \(S_{t}\) does not penalise inactive slots while preserving backward compatibility for historical ones.}}. Missed fills trigger a slashing penalty of \SI{10}{\percent} of current voting power.

\medskip

To make this precise, we model keeper selection as:
\begin{align*}
&\text{TriggerBot}(t) \in \mathbb{B} \text{ monitors } \texttt{isLiquidatablePosition}, \\
&\text{FillBot}_k(t) = \arg\max_{j \in[1,21]} \text{VP}_j(t) \quad \text{(ve-priority)}.
\end{align*}
Here, \( \mathbb{B} \) is the keeper bot set, and \( \text{VP}_j(t) = \text{stake}_j \cdot f(T_j) \) denotes voting power from ve-style staking, where \( f(T_j) \) increases with lock duration (e.g. \( f(T_j) = T_j / T_{\max} \)).

To reflect load-driven latency, expected fill time decays with active TVL:
\[
\mathbb{E}[\tau_{\text{fill}}] = 60\,\text{s} \cdot \exp\left(-\lambda \sum \text{TVL}_{\text{active}}\right).
\]
This extends the exclusivity model of \textcite{stableunit2025spec} with congestion-aware timing.


%=====================================================================%  
\section{Reserve price \& oracle integrity}
\label{sec:reserve}
%=====================================================================%  
To enhance resilience against manipulation and outages, we recommend periodically revisiting the selected oracle ensemble \parencite{deng2024safeguarding}. Let \(\mathcal{O}_t = \{O_{1}, O_{2}, \dots, O_{n}\}\) be the approved oracle set at time \(t\). Following \textcite{eskandari2021sok} we define the reserve price as a risk-adjusted median oracle:

\[
P_{\text{reserve}} = \operatorname*{median}_{O \in \mathcal{O}_t} \left[ P_{O} \right] \cdot (1 - \delta_{\text{VaR}, \alpha}),
\]
where \(\delta_{\text{VaR}, \alpha}\) is a Value-at-Risk (VaR)-derived haircut at confidence level \(\alpha \in (0,1)\).\footnote{Empirically, \(\delta_{\text{VaR}, 0.999} = 2.3\%\) for core crypto assets and \(5.7\%\) for long-tail tokens, based on the worst 0.1\% of historical oracle delay scenarios \parencite[Fig.~10]{tian2025defi}.}

\medskip

To formalise this:
\[
\delta_{\text{VaR}, \alpha} = \inf\left\{ d \in \mathbb{R}^+ : \Pr\left(L > d \cdot V\right) \leq 1 - \alpha \right\},
\]
where \( L \) is the liquidation loss random variable\footnote{The distribution of random liquidation loss \(L\), which depends on the current ETH price \(P_t\), must be rigorously verified. We propose to start with the assumption that \(P_t\) follows a jump-diffusion process \parencite{merton1976option}:
\[
d\ln P_t = \mu\, dt + \sigma\, dW_t + J\, dN_t,
\]
where \(W_t\) is standard Brownian motion, \(N_t\) is a Poisson process with intensity \(\lambda\), and \(J \sim \mathcal{N}(\mu_J, \sigma_J^2)\) represents log-jump magnitudes. The liquidation loss is then modeled as:
\[
L = \max\left(0, V - Q \cdot P_t e^{Z_T} \right),
\]
where \(Z_T\) is the total log return  (as defined in \secref{sec:stress}) over the liquidation horizon \(T\). The VaR-derived haircut \(\delta_{\text{VaR}, \alpha}\) is estimated numerically via Monte Carlo or saddlepoint approximation of the distribution of \(Z_T\) \parencite{kou2002jump}.}, and \( V \) is collateral value.\parencite{bcbs_marketrisk_2019}.\footnote{While the use of Value-at-Risk (VaR) is mandated in traditional finance under Basel III \parencite{bcbs_baselIII_2017}, its application in DeFi protocols remains experimental. Percentile-based liquidation risk model was proposed by \textcite{chainlink2025svr}, and \textcite{llamarisk2025svrreview} has independently applied it in the context of Aave's SVR integration proposal , which remains under community review.}

\medskip

The median operator is preferred for aggregation, as its deviation from the true price satisfies:
\[
\Pr\left(\left| P_{\text{median}} - P_{\text{true}} \right| > \epsilon\right) \leq 2\Phi\left(-\frac{\epsilon\sqrt{3}}{\sigma}\right),
\]
where \( \sigma \) denotes the standard deviation across oracle sources \parencite{eskandari2021sok}. This assumes independent, symmetric noise and facilitates zk-verifiable aggregation within secure oracle frameworks.


%=====================================================================%  
\section{Buyback and Debt Repayment}
\label{sec:buyback}
%=====================================================================%  
After a successful liquidation, received stablecoins are temporarily buffered in the \texttt{StableUnitBuyBack} module.  
Unlike immediate conversion heuristics, this module executes conversion and repayment only when reserve-price conditions are satisfied.

\begin{itemize}
  \item Accumulates whitelisted stablecoins from filled auctions;
  \item Repurchases USD Pro via AMM LPs or OTC, with near-zero slippage;
  \item Burns USD Pro to repay the corresponding CDP debt;
  \item Sends any surplus to a profit distribution contract.
\end{itemize}

This approach improves capital efficiency while reducing slippage.  
Future enhancements may include deploying protocol-owned liquidity (POL) on Balancer or Uniswap for deeper USD Pro markets.

%=====================================================================% 
\section{Keeper Incentives and Bounty Curve}
\label{sec:keeper}
%=====================================================================%  
Recent protocols \parencite{aave-liquidation-docs, compound-liquidation-docs, makerdao-auctions-docs} continue to rely on flat liquidation incentives. Simulated improvements to responsiveness via size-aware tips have been explored in \textcite{kirillov2022stablesims}.

In this work, to align protocol costs more closely with execution realities, we propose a size-sensitive marginal bounty schedule:
\[
\text{bounty rate}(V)=\min\!\left[\;\beta,\;\frac{\alpha}{\sqrt{V}}\right],
\qquad V=\text{vault value in \$}.
\]
This formulation preserves strong incentives for small vaults while naturally tapering payouts on large positions, challenging the historical skew toward whales in favor of strengthening protocol reserves.

Governance sets:
\begin{itemize}[noitemsep,topsep=0pt]
  \item \( \alpha \): the reference bounty factor (e.g., calibrated so that \( V_0 = \$50,000 \) yields 0.1\%),
  \item \( \beta \): a hard cap (e.g., 0.25\%) to prevent runaway APRs on micro-vaults.
\end{itemize}

This form ensures:
\begin{itemize}[noitemsep,topsep=0pt]
  \item Bounties grow sublinearly with vault size.
  \item Marginal outflow to keepers falls as \( 1/\sqrt{V} \).
  \item Rewards remain competitive across vault sizes without overpaying on large ones.
\end{itemize}

\paragraph{Implementation.} The curve can be implemented in the keeper module as:

\begin{lstlisting}[caption={Reward curve for vault liquidations}]
function _bountyRate(uint256 vaultUsd) internal view returns (uint256) {
    uint256 rate = alphaRay / sqrt(vaultUsd);
    return rate > betaRay ? betaRay : rate;
}
\end{lstlisting}

Empirical benchmarks and governance calibration are discussed in the Appendix.


%=====================================================================%  
\section{Cross-protocol contagion defence}
\label{sec:contagion}
%=====================================================================%  
Signed JSON heat-maps report live liquidation pressure across interconnected protocols.  To prevent feedback loops, peers may impose a \SIrange{60}{120}{\second} cooldown period.  

Systemic criticality for each node (e.g., a vault, position, or protocol address) is quantified using graph-based centrality.  
One option is $k$-shell decomposition \parencite{battiston2012debtrank}, which identifies the structural core of the graph but incurs cubic complexity $\mathcal{O}(n^3)$ in the number of nodes $n$.  
As a scalable alternative, criticality can be approximated using PageRank centrality \parencite{page1999pagerank}; the algorithm converges in $\mathcal{O}(m)$ per iteration, where $m$ is the number of edges in the liquidation correlation graph

If the cumulative value of outer-shell nodes (i.e., low-degree participants) exceeds \SI{5}{\percent} of the protocol’s total value locked (TVL), automatic circuit breakers are triggered to pause liquidations.

%=====================================================================%
\section{Governance}
\label{sec:governance}
%=====================================================================%
The proposed architecture delegates control of key risk parameters \((\kappa, Y, \psi, \delta)\) to an on-chain governance module, \texttt{RiskConfig}, which is secured by a 7-day timelock to allow community review and prevent instant changes.

To ensure transparency and auditability, all stress-test notebooks and VaR analyses are permanently stored using decentralized, content-addressed systems such as Arweave or IPFS.\footnote{Risk analyses and stress-test notebooks are pinned using decentralized storage: Arweave for permanent ledger-backed publishing \parencite{williams2023arweave}, and IPFS for content-addressable versioning \parencite{benet2014ipfs}.}


%=====================================================================%  
\section{Stress test: ETH flash crash 11 Dec 2024}
%=====================================================================%  

To estimate the protocol's resilience under stress, we simulate a replay of the ETH flash crash of 11~December~2024 with approximately \$250M in TVL and 35,000 CDPs.
  
Table~\ref{tab:stress} compares liquidation outcomes against a baseline fixed-spread engine.

\begin{table}[H]
\centering
\caption{BoC replay (\$250 M TVL, 35 000 CDPs).}
\label{tab:stress}
\begin{tabular}{l
                S[table-format=-2.1]
                S[table-format=-2.1]
                S[table-format=+2.1]}
\toprule
{Metric} &
{\textsc{5\% spread}} &
{\textsc{Proposed}} &
{Improvement\,/\,pp}\\
\midrule
Peak 5-min impact (\si{\percent}) & -14.2 & -3.8 & +10.4 \\
Bad debt / TVL (\si{\percent})    &  13.6 &  2.1 & -84.6 \\
Protocols hit                     &  8    &  1   & -87.5 \\
Clearance time                    & 50 s  & 6 min & — \\
\bottomrule
\end{tabular}
\end{table}


%=====================================================================%  
\section{Conclusion}
%=====================================================================%  
This work proposes a cascade-resilient liquidation architecture grounded in mathematically sound mechanisms—including stress detection, liquidity-aware tranching, VaR-based reserve pricing, and sealed-bid auctions. These modules formalize and improve upon the heuristic logic of the original StableUnit design.

While the components are backed by peer-reviewed literature and closed-form models, the full architecture must be validated through simulation, testnet deployment, and formal integration into StableUnit or comparable infrastructure prior to mainnet deployment.

%=====================================================================% 
% --------------------  APPENDIX --------------------
\clearpage
\appendix
%\addcontentsline{toc}{section}{Appendix}

%==================================================================% 
\section*{Appendix A: Stochastic Liquidation Framework — Proof of Solvency and $\lambda$ Calibration}

\addcontentsline{toc}{section}{Appendix}
\subsection{Tier–1 Execution Model (Privileged Slot)}
\label{subsec:tier1}
%======================================================================

Throughout we fix a filtered probability space
$(\Omega,\mathcal F,(\mathcal F_s)_{s\ge0},\mathbb P)$ satisfying the
usual conditions (right continuity and completeness).
All random variables are assumed $(\mathcal F_s)$–adapted unless
explicitly declared otherwise.

%----------------------------------------------------------------------
\paragraph{Deterministic stopping-time horizon.}
At a liquidation decision time $t\!\ge0$ governance
records\footnote{%
`TVL' is total value locked in the protocol at time $t$.  Because TVL
is updated on chain, we model it as an $\mathcal F_t$–measurable
random variable.}
\(
   \mathrm{TVL}_t\in[0,\infty).
\)
Given a scale parameter
$\lambda\!\in[\lambda_{\min},\lambda_{\max}]$ the Tier-1 latency
budget is declared
\[
   \boxed{%
     \tau_1(\lambda)
       := 60 \text{ s}\,
          \exp\!\bigl(-\lambda\,\mathrm{TVL}_t\bigr)
   }.
\]
Because $\mathrm{TVL}_t$ is $\mathcal F_t$–measurable,
$\tau_1(\lambda)$ is \emph{$\mathcal F_t$–measurable}, hence
deterministic once $\mathcal F_t$ is fixed.  Set
\(
   T := t+\tau_1(\lambda);
\)
$T$ is an $\mathcal F$–stopping time
(see \textcite[Ex.\ 3.3.3]{karatzas1991}).

%----------------------------------------------------------------------
\paragraph{Privileged-keeper primitives (modelling assumption).}
Conditioned on $\mathcal F_t$, the following random variables are
\emph{independent}:

\[
\begin{aligned}
I_{\mathrm{on}}
   &\sim \mathrm{Bernoulli}(p)
   &&\text{(keeper uptime, \textcite{li2000});}\\[4pt]
V_{\mathrm{priv}}
   &\sim \mathrm{Log\mathcal N}(\mu_v,\sigma_v^2)
   &&\text{(slot valuation, \textcite{pai2024});}\\[4pt]
C_{\mathrm{priv}}
   &\sim \mathrm{Pareto}(x_m,\alpha)
   &&\text{(capital tail, \textcite{celig2025});}\\[4pt]
D_{\mathrm{lat}}
   &\sim \mathrm{Weibull}(k,\lambda)
   &&\text{(network latency, \textcite{szalachowski2019}).}
\end{aligned}
\]

Independence is a design choice:  
each variable is generated by a distinct economic or infrastructural
process (hardware reliability, private value, wallet wealth,
propagation lag).  No structural link is known ex-ante; the tail
dependencies are empirically negligible at the one-second horizon.

%----------------------------------------------------------------------
\paragraph{Success factors.}
Define
\[
\begin{aligned}
q_{\mathrm{on}}
    &:= \Pr[I_{\mathrm{on}}=1\mid\mathcal F_t],\\
q_{\mathrm{val}}
    &:= \Pr[V_{\mathrm{priv}}\ge P_r(t)\mid\mathcal F_t],\\
q_{\mathrm{cap}}
    &:= \Pr[C_{\mathrm{priv}}\ge Q_u(t)\mid\mathcal F_t],\\
q_{\mathrm{lat}}(\lambda)
    &:= \Pr[D_{\mathrm{lat}}\le\tau_1(\lambda)\mid\mathcal F_t],\\
q_1(\lambda)
    &:= q_{\mathrm{on}}\,
        q_{\mathrm{val}}\,
        q_{\mathrm{cap}}\,
        q_{\mathrm{lat}}(\lambda).
\end{aligned}
\]
Because the four events are $\mathcal F_t$–conditionally independent,
the product rule follows from the very
definition of independence (\textcite[Def.\ 9.1]{williams1991}).

%----------------------------------------------------------------------
\paragraph{Oracle price dynamics.}
The (ask) oracle price $(P_s)_{s\ge t}$ evolves as
\[
   \mathrm dP_s
      = \sigma_s P_s \,\mathrm dW_s,
   \qquad
   0\le\sigma_s\le\sigma_{\max}<\infty,
\]
with $(W_s)$ a standard $(\mathcal F_s)$–Brownian motion and
$\sigma_s$ progressively measurable.  The bounded-volatility
assumption is conservative—Ethereum’s 90‐second historical realised
volatility rarely exceeds $3\%$—and ensures sub-Gaussian concentration
bounds.

%----------------------------------------------------------------------
\paragraph{Trigger and clearing prices.}
The protocol uses a VaR haircut $\delta_{\mathrm{VaR}}\in(0,1)$.
\[
   P_r(t)
      := (1-\delta_{\mathrm{VaR}})\,P_t,
   \qquad
   P_{\mathrm{clear}}(t)
      := \max\{V_{\mathrm{priv}}, P_r(t)\}.
\]

%======================================================================
\begin{theorem}[Tier-1 completion and solvency]
\label{thm:tier1-completion}
With notation as above, the following hold $\mathbb P$-almost surely
on $\mathcal F_t$:

\begin{enumerate}
\item[(i)] \textbf{Exact fill probability}
      \[
         \Pr\bigl[Q_u(T)=0\mid\mathcal F_t\bigr] = q_1(\lambda).
      \]

\item[(ii)] \textbf{Deterministic price floor}
      \[
         \mathbf 1_{\{Q_u(T)=0\}}
            \;P_{\mathrm{clear}}(t)
         \;\ge\;
         \mathbf 1_{\{Q_u(T)=0\}}
            \;P_r(t).
      \]

\item[(iii)] \textbf{Exponential bad-debt tail}
      \[
         \Pr\!\Bigl[
             P_{\mathrm{clear}}(t)\,Q_u(t)<D_t
             \;\Bigm|\;\mathcal F_t
         \Bigr]
         \;\le\;
         2\exp\!\Bigl(
            -\,\frac{(\delta_{\mathrm{VaR}}P_t)^2}
                    {2\sigma_{\max}^2\,\tau_1(\lambda)}
         \Bigr).
      \]

\item[(iv)] \textbf{Escalation safety}
      \(
         \Pr[Q_u(T)>0\mid\mathcal F_t] = 1-q_1(\lambda).
      \)
      For that residual mass Tier-2 guarantees
      \(
        P_{\mathrm{clear}}\ge P_r(t)(1-\delta_{\max})
      \)
      by Theorem~\ref{thm:tier2_solvency}.
\end{enumerate}
\end{theorem}

%----------------------------------------------------------------------
\begin{proof}
\emph{(i) Exact fill probability.}
Write
\(
  A_1:=\{I_{\mathrm{on}}=1\},\;
  A_2:=\{V_{\mathrm{priv}}\ge P_r(t)\},\;
  A_3:=\{C_{\mathrm{priv}}\ge Q_u(t)\},\;
  A_4:=\{D_{\mathrm{lat}}\le\tau_1(\lambda)\}.
\)
Tier-1 fills iff
\(
  \mathbf 1_{\{Q_u(T)=0\}}=\prod_{i=1}^4\mathbf 1_{A_i}.
\)
Conditional independence gives
\(
  \Pr[\cap_{i=1}^4A_i\mid\mathcal F_t]
     =\prod_{i=1}^4\Pr[A_i\mid\mathcal F_t]
     =q_1(\lambda).
\)

\smallskip
\noindent
\emph{(ii) Deterministic price floor.}
Protocol rules forbid bids with
$V_{\mathrm{priv}}<P_r(t)$, so on $A_2$,
$P_{\mathrm{clear}}(t)=V_{\mathrm{priv}}\ge P_r(t)$; on $A_2^{\mathrm c}$
we already have $P_{\mathrm{clear}}(t)=P_r(t)$.

\smallskip
\noindent

\item[(iii)]  \emph{Bad-debt tail bound.}
      Define the log–return semimartingale
      \[
         X_s:=\log\frac{P_s}{P_t}
              =M_s-\tfrac12\langle M\rangle_s,
         \qquad
         M_s:=\int_t^s\sigma_u\,dW_u.
      \]
      Because \(0\le\sigma_u\le\sigma_{\max}\),
      the quadratic variation satisfies
      \(\langle M\rangle_{s\wedge T}
          \le\sigma_{\max}^2(s-t)\).
      Freedman’s continuous-time martingale inequality%
\footnote{%
\textbf{Derivation.}  Set
\(M_s=\int_t^s\sigma_u\,dW_u\)
so that
\(X_s=M_s-\frac12\langle M\rangle_s\).
Because \(0\!\le\!\sigma_u\!\le\!\sigma_{\max}\),
\(\langle M\rangle_{s\wedge T}\le\sigma_{\max}^2(s-t)\).
Freedman’s result \parencite[Th.\,2]{freedman1975}
gives
\(
  \Pr[\sup_{t\le s\le T}|M_s|\ge\varepsilon\mid\mathcal F_t]
  \le 2\exp(-\varepsilon^{2}/(2\sigma_{\max}^{2}\tau_1(\lambda))).
\)
Since \(|X_s|\ge|M_s|\), the same bound holds for \(X_T\).
If \(P_s\) is extended to a finite-activity
      jump–diffusion
      \(dP_s=\sigma_s P_s dW_s+P_{s-}\,dJ_s\)
      with compensator
      \(\nu(ds\,dz)\),
      replace Freedman by the exponential martingale inequality for
      semimartingales with bounded predictable quadratic variation
      \parencite[Th.\,8.7]{applebaum2009}.  One obtains the same
      exponential form with
      \(\sigma_{\max}^2\) augmented by
      \(\int_{t}^{T}\!\!\int_{\mathbb R}z^2\nu(ds\,dz)\).).
}\,
therefore gives, for any \(\varepsilon>0\),

      \[
        \mathbb P\!\Bigl[
            |X_T|\ge\varepsilon
            \,\bigm|\,\mathcal F_t
        \Bigr]
        \;\le\;
        2\exp\!\Bigl(
            -\frac{\varepsilon^{2}}
                   {2\sigma_{\max}^{2}\,\tau_1(\lambda)}
        \Bigr).
      \]
      Put \(\varepsilon:=\log(1+\delta_{\mathrm{VaR}})\); then
      \(|P_T/P_t-1|\ge\delta_{\mathrm{VaR}}\)
      implies \(|X_T|\ge\varepsilon\), whence
      \[
        \mathbb P\!\Bigl[
            |P_T-P_t|\ge\delta_{\mathrm{VaR}}P_t
            \,\bigm|\,\mathcal F_t
        \Bigr]
        \;\le\;
        2\exp\!\Bigl(
            -\frac{\log^{2}(1+\delta_{\mathrm{VaR}})}
                   {2\sigma_{\max}^{2}\,\tau_1(\lambda)}
        \Bigr).
      \]
      Since \(P_{\mathrm{clear}}\ge P_r(t)\) by (ii), the same
      bound upper-controls
      \(\mathbb P[P_{\mathrm{clear}}Q_u<D_t\mid\mathcal F_t]\).
      %
      

\smallskip
\noindent
\emph{(iv) Escalation probability.}
By construction
\(Q_u(T)=0\) iff all four $A_i$ occur.
Hence $\Pr[Q_u(T)>0\mid\mathcal F_t]=1-q_1(\lambda)$ and the
Tier-2 guarantee is external to Tier-1.
\end{proof}

%----------------------------------------------------------------------
%----------------------------------------------------------------------
\begin{corollary}[Governance tuning of $\lambda$]
\label{cor:tune-lambda}
Fix a target Tier-1 success probability $\beta\!\in\!(0,1)$ and recall
\[
   q_{\mathrm{fill}}(\lambda)
      := q_{\mathrm{on}}\,
         q_{\mathrm{val}}\,
         q_{\mathrm{cap}}\,
         q_{\mathrm{lat}}(\lambda).
\]
Because the Weibull survivor
\(q_{\mathrm{lat}}(\lambda)=1-\exp\!\bigl(-(\tau_1(\lambda)/\lambda)^k\bigr)\)
is \emph{strictly increasing} in $\lambda$ for every shape parameter
$k>0$, the map
\(\lambda\mapsto q_{\mathrm{fill}}(\lambda)\)
is strictly increasing as well.  Define the unique threshold
\[
   \boxed{%
     \lambda^{\star}
       := \inf\Bigl\{
              \lambda\in[\lambda_{\min},\lambda_{\max}]
              : q_{\mathrm{fill}}(\lambda)\ge\beta
            \Bigr\}.
   }
\]
Then any \(\lambda\ge\lambda^{\star}\) guarantees
\[
   \mathbb P\!\bigl[\text{Tier-1 fill}\mid\mathcal F_t\bigr]\;\ge\;\beta.
\]
\end{corollary}

\paragraph{Implementation note.}
On chain, $\lambda^{\star}$ can be found by a bisection search over
\([\lambda_{\min},\lambda_{\max}]\) because monotonicity ensures a
single crossing point.  Evaluate \(q_{\mathrm{lat}}(\lambda)\) with
the closed-form Weibull CDF and multiply by the on-chain estimates of
\(q_{\mathrm{on}},\,q_{\mathrm{val}},\,q_{\mathrm{cap}}\) obtained from
rolling keeper statistics.\footnote{Gas-efficient: only two 256-bit
exponentials and one multiplication per iteration.}


%----------------------------------------------------------
%  Central Solvency Theorem for the Tier-2 Sealed-Bid Auction
%----------------------------------------------------------
\clearpage
%\addcontentsline{toc}{section}{Appendix}

%==================================================================% 
\section*{Appendix B: Dutch Auction Convergence and Vickrey probability}
\addcontentsline{toc}{section}{Appendix B: Tier-2 Liquidation Guarantees and Dutch Auction Convergence}

\begin{lemma}[Dutch-Clock Convergence]\label{lem:dutch_clock}
Fix a reserve price \(P_{\mathrm{reserve}}\) and let the Dutch auction
tick sequence be
\[
  P_{\mathrm{clock}}(k)
  := P_{\mathrm{reserve}}
     \prod_{i=0}^{\,k-1}\!(1-\delta_i),
  \qquad
  0<\delta_i\le\delta_{\max}<1 .
\]
Then:
\begin{enumerate}
\item[\textup{(a)}] \(P_{\mathrm{clock}}(k)\) is strictly
      decreasing and bounded below by
      \(P_{\mathrm{reserve}}\,(1-\delta_{\max})^{k}\).  
\item[\textup{(b)}] The limit exists and satisfies
\[
  \lim_{k\to\infty} P_{\mathrm{clock}}(k)
  = L \in (0, P_{\mathrm{reserve}}].
\]
\end{enumerate}
\end{lemma}
\begin{proof}
We prove each part separately.

\textbf{(a)} \emph{Monotonicity and lower bound.}  
Each tick $\delta_i$ satisfies $0 < \delta_i \le \delta_{\max} < 1$, so each factor $(1 - \delta_i)$ lies strictly between 0 and 1.  
Therefore, the product
\[
P_{\mathrm{clock}}(k) = P_{\mathrm{reserve}} \cdot \prod_{i=0}^{k-1}(1 - \delta_i)
\]
is strictly decreasing in $k$.  
Moreover, since every $(1 - \delta_i) \ge (1 - \delta_{\max})$, we can bound the product below:
\[
\prod_{i=0}^{k-1}(1 - \delta_i) \ge (1 - \delta_{\max})^k.
\]
Multiplying both sides by $P_{\mathrm{reserve}}$ gives the claimed lower bound:
\[
P_{\mathrm{clock}}(k) \ge P_{\mathrm{reserve}} \cdot (1 - \delta_{\max})^k.
\]

\textbf{(b)} \emph{Convergence.}  
Since $P_{\mathrm{clock}}(k)$ is strictly decreasing and bounded below (by part (a)), the Monotone Convergence Theorem implies that the sequence has a finite limit:
\[
\lim_{k \to \infty} P_{\mathrm{clock}}(k) = L \in \left[0,\ P_{\mathrm{reserve}}\right].
\]
Moreover, the bound in part (a) shows that $L \ge 0$ with an explicit rate of decay:
\[
0 < P_{\mathrm{clock}}(k) - L \le P_{\mathrm{reserve}} \cdot (1 - \delta_{\max})^k.
\]
\end{proof}
\begin{theorem}[Tier-2 Solvency and Vickrey-Validity]
\label{thm:tier2_solvency}
Consider a sealed-bid auction with the following setup:
\begin{itemize}
    \item Let $N \sim \operatorname{Pois}(\mu)$ be the (random) number of keepers who observe the auction.
    \item Each keeper $i$ draws a private valuation $V_i$ (i.i.d.) and can submit a bid if $V_i \geq P_{\mathrm{reserve}}$, where $P_{\mathrm{reserve}}$ is the protocol-defined reserve price.
    \item Let $p := \Pr[V_i \geq P_{\mathrm{reserve}}] \in (0,1]$ and define $\theta := \mu p$.
    \item Let $N^*$ denote the number of valid bids submitted.
    \item Let $\delta_{\max}$ be the protocol-defined maximum haircut for the Dutch auction tier.
\end{itemize}

Then:
\begin{enumerate}
    \item[\textup{(i)}] \textbf{Bid-count distribution.}  
        The number of valid bids is distributed as $N^* \sim \operatorname{Pois}(\theta)$.

    \item[\textup{(ii)}] \textbf{Probability of a competitive (Vickrey) outcome.}  
        The probability that at least two valid bids are submitted is
        \[
            \Pr[N^* \geq 2] = 1 - e^{-\theta}(1 + \theta).
        \]

    \item[\textup{(iii)}] \textbf{Solvency guarantee.}  
        In all cases, the clearing price $P_{\mathrm{clear}}$ satisfies
        \[
            P_{\mathrm{clear}} \geq P_{\mathrm{reserve}} (1 - \delta_{\max}),
        \]
        so Tier-2 cannot clear at a price below the worst-case Dutch auction outcome, and bad debt is impossible at this stage.
\end{enumerate}
\end{theorem}

\begin{proof}
We proceed in three parts, corresponding to the theorem's statements.

\textbf{(i) Bid-count distribution.}  

Given $N \sim \operatorname{Pois}(\mu)$, each keeper independently submits a valid bid with probability $p$. By the classical Poisson thinning theorem (see \textcite[Thm.~5, §II.6]{feller1968}), the number of valid bids $N^*$ is itself Poisson distributed:
\[
    N^* \sim \operatorname{Pois}(\mu p) = \operatorname{Pois}(\theta).
\]

\textbf{(ii) Probability of a competitive (Vickrey) outcome.}  
A competitive (Vickrey) auction requires at least two valid bids. For $N^* \sim \operatorname{Pois}(\theta)$, the probability of fewer than two valid bids is
\[
    \Pr[N^* < 2] = \Pr[N^* = 0] + \Pr[N^* = 1] = e^{-\theta} + \theta e^{-\theta}.
\]
Therefore, the probability of at least two valid bids is
\[
    \Pr[N^* \geq 2] = 1 - e^{-\theta}(1 + \theta).
\]

\textbf{(iii) Solvency guarantee.}  
We consider all possible cases for $N^*$:
\begin{itemize}
    \item \emph{Case 1: $N^* \geq 2$.}  
        The auction is resolved as a Vickrey (second-price) auction. The clearing price is the second-highest valid bid, which by construction is at least $P_{\mathrm{reserve}}$:
        \[
            P_{\mathrm{clear}} \geq P_{\mathrm{reserve}}.
        \]
    \item \emph{Case 2: $N^* = 1$.}  
        By protocol rule, if only one valid bid is submitted, the clearing price is set to the reserve price:
        \[
            P_{\mathrm{clear}} = P_{\mathrm{reserve}}.
        \]
    \item \emph{Case 3: $N^* = 0$.}  
        If no valid bids are submitted, the tranche escalates to the Dutch auction tier. By Lemma~4.1 (Monotone Clock Convergence), the Dutch auction cannot clear below $P_{\mathrm{reserve}}(1-\delta_{\max})$:
        \[
            P_{\mathrm{clear}} \geq P_{\mathrm{reserve}}(1 - \delta_{\max}).
        \]
\end{itemize}
In all cases, the clearing price meets or exceeds $P_{\mathrm{reserve}}(1 - \delta_{\max})$, ensuring that Tier-2 cannot result in a price lower than the worst-case Dutch auction outcome. Thus, bad debt is impossible at this stage.

\clearpage
\section*{Appendix C: Vickrey Upgrade Layer for MakerDAO Auctions}
\addcontentsline{toc}{section}{Appendix C: Vickrey Upgrade Layer for MakerDAO Auctions}

This appendix provides a compile-ready version of the \texttt{ClipVickrey} contract, which wraps MakerDAO's native Dutch-auction engine (\texttt{Clip.sol}) \parencite{makerdao-clip-docs} to implement a sealed-bid, second-price (Vickrey) auction layer. Crucially, this is an \emph{additive upgrade}:
\begin{itemize}
  \item \texttt{ClipVickrey} owns no collateral and does not replace \texttt{Clip.sol}.
  \item It orchestrates commit-reveal logic and delegates final settlement via \texttt{clip.\_settle()}.
  \item If fewer than two valid bids are revealed, the system falls back to the standard Dutch price curve.
\end{itemize}
This upgrade model ensures safety: the original \texttt{Clip} logic remains untouched and audited. At worst, the auction behaves exactly as before. Timeouts are constructor-tunable and allow governance to balance latency with bid competition.

\bigskip

\noindent\textbf{Vickrey Wrapper:} \texttt{ClipVickrey.sol}

\begin{minted}[fontsize=\small, bgcolor=codegray, linenos]{solidity}
// SPDX-License-Identifier: AGPL-3.0
pragma solidity ^0.8.20;

interface ClipLike {
    function price() external view returns (uint256);
    function _settle(address buyer, uint256 price) external;
}

contract ClipVickrey {
    ClipLike public immutable clip;

    enum Phase { Commit, Reveal, Cleared }
    Phase public phase;
    uint48 public commitDuration;
    uint48 public revealDuration;
    uint64 public commitEnd;
    uint64 public revealEnd;

    struct Bid {
        bytes32 commitHash;
        uint256 amount;
        bool revealed;
    }
    mapping(address => Bid) public bids;
    address public highestBidder;
    uint256 public highestBid;
    uint256 public secondBid;

    event BidCommitted(address indexed bidder);
    event BidRevealed(address indexed bidder, uint256 amount);
    event AuctionSettled(address winner, uint256 clearingPrice, bool vickrey);

    constructor(address _clip, uint48 _commitDuration, uint48 _revealDuration) {
        require(_clip != address(0), "ClipVickrey/null-clip");
        require(_commitDuration > 0 && _revealDuration > 0, "ClipVickrey/bad-durations");

        clip = ClipLike(_clip);
        commitDuration = _commitDuration;
        revealDuration = _revealDuration;
        commitEnd = uint64(block.timestamp + _commitDuration);
        revealEnd = uint64(commitEnd + _revealDuration);
        phase = Phase.Commit;
    }

    function commitBid(bytes32 hash) external {
        require(phase == Phase.Commit, "commit/phase-over");
        require(block.timestamp < commitEnd, "commit/window-closed");
        require(bids[msg.sender].commitHash == bytes32(0), "commit/duplicate");

        bids[msg.sender].commitHash = hash;
        emit BidCommitted(msg.sender);
    }

    function revealBid(uint256 amount, bytes32 salt) external {
        require(block.timestamp >= commitEnd, "reveal/not-started");
        require(block.timestamp < revealEnd, "reveal/ended");
        require(phase != Phase.Cleared, "reveal/already-cleared");

        Bid storage b = bids[msg.sender];
        require(b.commitHash != bytes32(0), "reveal/no-commit");
        require(!b.revealed, "reveal/dup");
        require(b.commitHash == keccak256(abi.encode(amount, salt)), "reveal/hash-mismatch");

        b.revealed = true;
        b.amount = amount;

        if (amount > highestBid) {
            secondBid = highestBid;
            highestBid = amount;
            highestBidder = msg.sender;
        } else if (amount > secondBid) {
            secondBid = amount;
        }

        phase = Phase.Reveal;
        emit BidRevealed(msg.sender, amount);
    }

    function finalize() external {
        require(block.timestamp >= revealEnd, "finalize/reveal-active");
        require(phase != Phase.Cleared, "finalize/done");

        bool vickrey = (highestBid > 0 && secondBid > 0);
        uint256 clearingPrice = vickrey ? secondBid : clip.price();
        address winner = vickrey ? highestBidder : msg.sender;

        clip._settle(winner, clearingPrice);
        phase = Phase.Cleared;

        emit AuctionSettled(winner, clearingPrice, vickrey);
    }

    function commitActive() external view returns (bool) {
        return (phase == Phase.Commit && block.timestamp < commitEnd);
    }

    function revealActive() external view returns (bool) {
        return (phase != Phase.Cleared &&
                block.timestamp >= commitEnd &&
                block.timestamp < revealEnd);
    }
}
\end{minted}

\bigskip

\noindent\textbf{Deployment Note:}  
After deploying \texttt{ClipVickrey}, governance must call \texttt{clip.rely(clipVickrey)} to authorize it. Keepers then follow:
\[
\texttt{commitBid()} \rightarrow \texttt{revealBid()} \rightarrow \texttt{finalize()}
\]
to settle via the Vickrey mechanism or fall back to Dutch clearing if no competition 

%=====================================================================%  
% --------------------  BIBLIOGRAPHY --------------------
\clearpage
\printbibliography
%=====================================================================%  
\end{document}
